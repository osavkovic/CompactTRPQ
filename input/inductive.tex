
\section{Inductive representation}
\label{sec:inductive}


\subsection{$\evalcitdbe{q}$}

\subsubsection{Definition}
\label{sec:evalbe_def}

\subsubsection{Correctness}
\label{sec:evalbe_correct}

\para{$\path_1/\path_2$}
Let $\u_1 = \tup{o_1, o_2, \tau_1, \delta_1}$ and $\u_2 = \tup{o_3, o_4, \tau_2, \delta_2}$,
with $o_2 = o_3$.

And let  $\u_1 \tjoin \u_2 = \tup{o_1, o_4, \tau'_1, \delta_1 + \delta_2, b,e}$,
with
\begin{align*}
\tau'_1 &=  (((\tau_1 + \delta_1) \cap \tau_2) \ominus \delta_1) \cap \tau_1\\
  b &= \max(b_1, b_2 - b_{\delta_1})\\
  e &= \min(e_1, e_2 - e_{\delta_1})
\end{align*}



For each $i \in \{1,2\}$ and $t \in \tau_i$,
we use $\delta_i(t)$ for the interval
\[\ld{\delta_{i}}\ b_{\delta_i} + \max(0, b_i - t) \ ,\ e_{\delta_i} - \max(0, t - e_i)\ \rd{\delta_{i}}\]
And similarly to what we did for $\tuplestd$,
we use $R_i$ for be the binary relation over $\td$ specified by the time points and distances in $\u_i$,
i.e.~$R_i = \{(t,t + d) \mid t \in \tau_i, d \in \delta_i(t)\}$.

Then the intervals in the set $\u_1 \tjoin \u_2$ should intuitively represent this relation $R_1 \join R_2$, i.e.

with
\begin{align*}
\tau'_1 &=  (((\tau_1 + \delta_1) \cap \tau_2) \ominus \delta_1) \cap \tau_1\\
  b &= \max(b_1, b_2 - b_{\delta_1})\\
  e &= \min(e_1, e_2 - e_{\delta_1})
%   \begin{cases}
%     b_1 &\te{if} b_2 \le b_1 + b_{\delta_1}\\
%     b_2 - b_{\delta_1} &\te{otherwise}
%   \end{cases}
% &
 % e =
 %  \begin{cases}
 %    e_1 &\te{if} e_1 + e_{\delta_1} \le e_2 \\
 %    e_2 - e_{\delta_1} &\te{otherwise}
 %  \end{cases}
\end{align*}


The proof that 

Now let $t \in $



%%% Local Variables:
%%% mode: latex
%%% TeX-master: "../appendix"
%%% End:
